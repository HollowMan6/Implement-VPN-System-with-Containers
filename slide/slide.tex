\documentclass{beamer}
\usepackage{blindtext}
\usepackage{graphicx}
\usepackage[SCI]{aaltologo}
\usetheme[invariant]{Aalto}
\setbeamertemplate{sections/subsections in toc}[square]

\title{Implementing a Virtual Network\\ System among Containers}
\author{\texorpdfstring{Songlin Jiang\\ \url{songlin.jiang@aalto.fi}}{Songlin Jiang}}
\institute{Tutor: Tuomas Aura}
\date{April 21, 2023}

\begin{document}
\frame{\titlepage}

\section*{Introduction}
\begin{frame}{Introduction}
\begin{itemize}
\item Virtual machines (VM) are commonly used for \textbf{building and testing} network system configurations.
\item  \textbf{containers} can replace \textbf{VMs} to overcome VMs' drawbacks. However, \textbf{no one has tried} to realize that using \textbf{Docker}.
\item Rise of \textbf{container technologies} has attracted great attention.
\end{itemize}
\begin{figure}[t!]
  \begin{center}
    \includegraphics[scale=0.7]{slide/img/vs.png}
    \label{fig:demo}
  \end{center}
\end{figure}
\end{frame}

\section{Docker Networking System}
\begin{frame}{Overview}
    \tableofcontents[currentsection]
\end{frame}
\subsection{Network Drivers}
\begin{frame}{Network Drivers}
\begin{itemize}
\item Docker employs a pluggable networking subsystem.
\item Default drivers for Docker networking system include the \textbf{bridge, host, overlay, IPVLAN, MACVLAN, and none}.
\item Possible candidates for implementing the virtual network system on one host machine are \textbf{bridge, IPVLAN, and MACVLAN}.
\begin{itemize}
    \item The \textbf{none} driver disables all networking.
    \item The \textbf{host} driver shares the same network stack with the host.
    \item The \textbf{overlay} driver creates a distributed network system among multiple Docker daemon hosts.
\end{itemize}
\end{itemize}
\end{frame}

\begin{frame}{MACVLAN}
We choose \textbf{MACVLAN} finally:
\begin{table}[t!]
  \begin{center}
    \caption{Network Drivers Comparison}
    \begin{tabular}{|l|c|c|c|}
    \hline
    \textbf{Items} & \textbf{bridge} & \textbf{IPVLAN} & \textbf{MACVLAN} \\
    \hline
    \textbf{Resources} & High & Low & Lowest \\
    \hline
    \textbf{MAC Address} & Different & Same & Different \\
    \hline
    \textbf{VM Migration} & No & No & Yes \\
    \hline
    \end{tabular}
    \label{tab:comparedrivers}
  \end{center}
\end{table}
\begin{itemize}
 \item MACVLAN is a trivial bridge that does not need to learn as it already knows every \textbf{MAC address} it can receive.
\item MACVLAN also supports \textbf{address configuration} and \textbf{discovery protocols}, as well as other \textbf{multicast protocols}.
\end{itemize}
\end{frame}

\subsection{Routing, Firewall and IPv6}
\begin{frame}{Routing, Firewall and IPv6}
\begin{itemize}
\item By default, Docker does not allow manipulating container network devices and setting routing tables or firewalls inside containers.
\item However, \textbf{net\_admin} capability in Linux allows us to:
\begin{enumerate}
\item \textbf{Make interface configuration.}
\item \textbf{Administrate IP firewall, masquerading, and accounting.}
\item \textbf{Modify routing tables.}
\item Bind to any address for transparent proxying.
\item Set the type-of-service (TOS).
\item Clear driver statistics.
\item Set the promiscuous mode.
\item Enable multicasting.
\item Use setsockopt(2) to set several socket options.
\end{enumerate}
\item \textbf{IPv6} is also supported in Docker containers.
\end{itemize}
\end{frame}

\section{Case Study: VPN}
\begin{frame}{Overview}
    \tableofcontents[currentsection]
\end{frame}
\subsection{VPN Overview}
\begin{frame}{VPN Overview}
\begin{itemize}
\item Simulate IoT devices in \textbf{sites A, B} connect to server in \textbf{cloud S}.
\begin{itemize}
    \item \textbf{Site A, B}, and \textbf{cloud S} both use the private IP addresses to improve security and save the IPv4 addresses.
    \item \textbf{Gateways A, B, S} connect \textbf{sites A, B, S} to the public Internet.
    \item The \textbf{router} represents the Internet between the sites and cloud.
    \item The address space between the \textbf{gateway} and \textbf{router} simulates public, routable IPv4 addresses, although they are all private.
    \item \textbf{Site A, B}, and \textbf{cloud S} use the \textbf{router} to access the Internet.
\end{itemize}
\item We experiment with 2 types of VPN using strongSwan (IPsec):
\begin{enumerate}
    \item \textbf{Site to Site}
    \item \textbf{Host to Host}
\end{enumerate}
\end{itemize}
\end{frame}

\subsection{Host to Host}
\begin{frame}{Host to Host}
\begin{figure}[t!]
  \begin{center}
    \includegraphics[scale=1]{figures/host-to-host.pdf}
    \label{fig:host2host}
  \end{center}
\end{figure}
\end{frame}

\subsection{Site to Site in IPv6}
\begin{frame}{Site to Site in IPv6}
\begin{figure}[t!]
  \begin{center}
    \includegraphics[scale=0.95]{figures/ipv6-site-to-site.pdf}
    \label{fig:ipv6site2site}
  \end{center}
\end{figure}
\end{frame}

\section{Evaluation}
\begin{frame}{Overview}
    \tableofcontents[currentsection]
\end{frame}
\subsection{Usability and Portability}
\begin{frame}{Usability and Portability}
With \textbf{Docker Compose}, the Docker orchestration tool, containers \textbf{outperforms} VirtualBox-based Vagrant in many aspects:
\begin{table}[t!]
  \begin{center}
    \caption{Functionalities Comparison}
    \begin{tabular}{|l|c|c|}
    \hline
    \textbf{Items} & \textbf{Vagrant + VirtualBox} & \textbf{Docker Compose} \\
    \hline
    \textbf{Resources} & Heavy & Lightweight \\
    \hline
    \textbf{Kernel} & Own & Shared (Namespaces)\\
    \hline
    \textbf{Scalability} & Hard & Easy \\
    \hline
    \textbf{M1/M2 Support} & Limited & Fully \\
    \hline
    \textbf{Image Hub} & Unavailable & Available (Docker Hub) \\
    \hline
    \textbf{Seamless} & No & Yes \\
    \hline
    \end{tabular}
    \label{tab:compare}
  \end{center}
\end{table}
\end{frame}

\subsection{Performance}
\begin{frame}{Performance}
    Table below shows the metrics for the performance evaluation:
    \begin{table}[t!]
      \begin{center}
        \caption{Performance Test Result in Average}
        \begin{tabular}{|l|lr|}
        \hline
        Solution               & Boot Time$^1$ & Memory$^2$ \\
        \hline
        Docker Compose         &   75 s    &  278 MB \\
        Vagrant + VirtualBox   &  689 s    &  4.5 GB \\
        \hline
        \end{tabular}
        \label{tab:result}
      \end{center}
    \end{table}
    The container based solution \textit{reduces}:
    \begin{enumerate}
    \item \textbf{Fresh boot time} by nearly \textit{90\%}
    \item \textbf{Memory consumption} by nearly \textit{94\%}
    \end{enumerate}
\end{frame}

\subsection{Security and Limitations}
\begin{frame}{Security and Limitations}
\begin{itemize}
\item The container network stacks are \textbf{isolated} from host.
\end{itemize}

\begin{itemize}
\item Limitations of the Docker networking model include:
\begin{enumerate}
\item Disallowing to assign \textbf{overlapped IP address ranges} by Docker, even for network interfaces that won't directly connected.
\item Disallowing to assign \textbf{IP address ending in ".1"} by Docker, as these addresses are reserved by Docker for gateways or routers.
\end{enumerate}
\item However, we can always choose to configure the IP addresses manually inside containers to bypass these limitations.
\end{itemize}
\end{frame}

\section{Conclusion}
\begin{frame}{Overview}
    \tableofcontents[currentsection]
\end{frame}
\begin{frame}{Conclusion}
\begin{itemize}
\item \textbf{It's possible to have a virtual network system with Docker!}
\item Containers are much more \textbf{lightweight} than virtual machines.
\item Containers \textbf{realized all functionalities} for VPN case study.
\item Container is a suitable \textbf{replacement} for virtual machines.
\end{itemize}
\begin{figure}[t!]
  \begin{center}
    \includegraphics[scale=0.25]{slide/img/docker-win.png}
    \label{fig:docker-win}
  \end{center}
\end{figure}
\end{frame}

\section{Q\&A}
\begin{frame}{Q\&A}
\centering
\huge{\textbf{Thanks for listening!}}

Any Questions?
\begin{figure}[t!]
  \begin{center}
    \includegraphics[scale=0.075]{slide/img/source.png}
    \caption{Scan to get the case study implementation source code}
    \label{fig:source}
  \end{center}
\end{figure}
\end{frame}
\end{document}
